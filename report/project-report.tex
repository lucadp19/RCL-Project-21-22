\documentclass[
    oneside,
    10pt,
    language=italian,
    a4paper,
    article
]{notes}
\usepackage{code}
\usepackage{minted}
\usemintedstyle{manni}

\renewcommand{\thesection}{\arabic{section}}

\definecolor{bg}{rgb}{0.97,0.97,0.97}

\setminted{
    bgcolor=bg,
    fontsize=\small,
    breaklines=true,
    escapeinside=||,
    mathescape=true,
}
\setmintedinline{
    fontsize=\normalsize
}
\newmintinline[injava]{java}{}
\newmintinline[make]{text}{}
\newmintinline[shell]{text}{}

\author{Luca De Paulis}
\title{Relazione sul Progetto di Reti di Calcolatori e Laboratorio \\
    \large A.A. 2021/2022 }

\begin{document}
\maketitle

\section{Introduzione e informazioni generali}
Il progetto di Reti di Calcolatori e Laboratorio dell'Anno Accademico 2021/2022
consiste nella realizzazione di un Social Network chiamato \sstrong{Winsome}.
Il Social Network è basato su un'architettura di tipo \emph{client-server}:
le due parti comunicano attraverso la rete grazie ad un'\emph{API}, che permette
al client di interfacciarsi con l'applicativo del server.

Il progetto è stato realizzato in Java, versione 17.

\section{Compilazione ed esecuzione}
Il progetto è interamente compilabile ed eseguibile attraverso l'uso dei comandi
\shell{javac} e \shell{java}; tuttavia, per semplificare il processo, è stato
incluso un \shell{Makefile} contenente alcune semplici regole.
\begin{description}
    \item[\make{make}] La regola di default compila il progetto: in particolare 
        crea le directory necessarie al corretto funzionamento degli applicativi,
        compila il codice sorgente e ne crea due file \shell{.jar}, che vengono
        salvati nella directory \shell{bin/}.
    \item[\make{make run-default-server}] Questa regola, se lanciata dopo la
        compilazione, esegue il server usando come parametri (ovvero come 
        \emph{path} del file di configurazione e della directory di logging, 
        cf. \Cref{sec:server}) i valori di default.
    \item[\make{make run-default-client}] Analogamente, esegue il client usando
        i parametri di default (in questo caso solamente il \emph{path} del
        file di configurazione, cf. \Cref{sec:client}).  
    \item[\make{make clean}] Rimuove i file \shell{.class}, i file \shell{.jar}
        e i file di log del server contenuti nella directory \shell{logs/}. 
\end{description}

\end{document}