\documentclass[
    oneside,
    10pt,
    language=italian,
    a4paper,
    article
]{notes}
\usepackage{code}
\usepackage{minted}
\usemintedstyle{manni}

\renewcommand{\thesection}{\arabic{section}}

\definecolor{bg}{rgb}{0.97,0.97,0.97}

\setminted{
    bgcolor=bg,
    fontsize=\small,
    breaklines=true,
    escapeinside=||,
    mathescape=true,
}
\setmintedinline{
    fontsize=\normalsize
}
\newmintinline[injava]{java}{}
\newmintinline[mono]{text}{}
\newmintinline[shell]{shell-session}{}

\author{Luca De Paulis}
\title{Relazione sul Progetto di Reti di Calcolatori e Laboratorio \\
    \large A.A. 2021/2022 }

\begin{document}
\maketitle

\section{Introduzione e informazioni generali}
Il progetto di Reti di Calcolatori e Laboratorio dell'Anno Accademico 2021/2022
consiste nella realizzazione di un Social Network chiamato \sstrong{Winsome}.
Il Social Network è basato su un'architettura di tipo \emph{client-server}:
le due parti comunicano attraverso la rete grazie ad un'API, che permette
al client di interfacciarsi con l'applicativo del server.

Il progetto è stato realizzato in Java, versione 17 ed è stato testato in ambiente
Linux.

\section{Compilazione ed esecuzione}
Il progetto è interamente compilabile ed eseguibile attraverso l'uso dei comandi
\mono{javac} e \mono{java}; tuttavia, per semplificare il processo, è stato
incluso un \mono{Makefile} contenente alcune semplici regole.
\begin{description}
    \item[\shell{$ make}] La regola di default compila il progetto: in particolare 
        crea le directory necessarie al corretto funzionamento degli applicativi,
        compila il codice sorgente e ne crea due file \mono{.jar}, che vengono
        salvati nella directory \mono{bin/}.
    \item[\shell{$ make run-default-server}] Questa regola, se lanciata dopo la
        compilazione, esegue il server usando come parametri (ovvero come 
        \emph{path} del file di configurazione e della directory di logging, 
        cf. \Cref{sec:server}) i valori di default.
    \item[\shell{$ make run-default-client}] Analogamente, esegue il client usando
        i parametri di default (in questo caso solamente il \emph{path} del
        file di configurazione, cf. \Cref{sec:client}).  
    \item[\shell{$ make clean}] Rimuove i file \mono{.class}, i file \mono{.jar}
        e i file di log del server contenuti nella directory \mono{logs/}. 
\end{description}

Per eseguire il client o il server passando dei parametri a riga di comando si
possono usare gli script Bash \mono{run-server.sh} e \mono{run-client.sh},
entrambi contenuti nella directory \mono{scripts/}. Per far ciò è necessario
innanzitutto renderli eseguibili, tramite ad esempio 
\shell{$ chmod a+x scripts/run-server.sh scripts/run-client.sh}, e poi eseguirli
normalmente come \shell{$ ./run-server.sh [...]}.

L'unica dipendenza del progetto è la libreria GSON (\mono{gson-2.8.6.jar}),
che si trova nella directory \mono{lib/}.

\section{Architettura generale del progetto} \label{sec:arch}
Il progetto è diviso in tre moduli principali: \begin{itemize}
    \item il \sstrong{server}, implementato nel package \injava{winsome.server}
        e avente come \emph{entry point} la classe 
        \injava{winsome.server.WinsomeServerMain}: esso viene raccolto
        nel file \mono{bin/winsome-server.jar};
    \item il \sstrong{client}, implementato nel package \injava{winsome.client}
        e avente come \emph{entry point} la classe
        \injava{winsome.client.WinsomeClientMain}: viene compresso nel
        file \mono{bin/winsome-client.jar}
    \item le \sstrong{API}, implementate nel package \injava{winsome.api}:
        insieme alle funzionalità di \emph{utility} del package 
        \injava{winsome.utils} le API vengono compresse nel file
        \mono{lib/winsome-api.jar}.
\end{itemize}

La scelta di separare le API dall'applicativo client ha lo scopo di separare il
più possibile i vari moduli: così facendo si lascia aperta la possibilità
di creare un client diverso che riusi le API fornite per interfacciarsi con
il server.
Inoltre sia il client che il server dipendono dalla libreria delle API per 
realizzare alcune funzionalità (ad esempio il parsing dei file di configurazione).

\section{Il package \injava{winsome.utils}}
I tre moduli descritti nella \Cref{sec:arch} dipendono da alcune funzionalità
messe a disposizione dal package di utilities:
\begin{itemize}
    \item il file \injava{winsome.utils.ConsoleColors} contiene i codici ASCII
        per usare i colori nei terminali: viene usato soprattutto dal client
        nella propria Command Line Interface;
    \item il package \injava{winsome.utils.cryptography} contiene la classe
        \injava{Hash}, che permette l'hashing di stringhe (utile per
        inviare e memorizzare le password), e l'eccezione 
        \injava{FailedHashException} per indicare il fallimento di un tentativo
        di hashing;
    \item il package \injava{winsome.utils.configs} contiene le classi
        \injava{AbstractConfig} e \injava{ConfigEntry}: esse consentono di parsare
        file nel formato descritto nella \Cref{sec:config}.
\end{itemize}

\section{I file di configurazione} \label{sec:config}
Sia il client che il server necessitano di un file di configurazione per 
inizializzare i vari parametri. La sintassi dei due file è la stessa, e corrisponde
alla sintassi base del linguaggio di markup YAML:
\begin{minted}{yaml}
    # comments and empty lines are allowed
    key: value # comments after values are allowed
\end{minted}
Due esempi di file di configurazione sono presenti nella directory \mono{configs/}
e di default, a meno di non specificare altrimenti negli argomenti a riga di
comando del client/server, tali file sono usati come file di configurazione.  

I campi dei file di configurazione devono essere tutti presenti una e una sola 
volta: in caso contrario il sistema segnala l'errore.

\subsection{File di configurazione del server}
Il file di configurazione del server ha i seguenti campi:
\begin{description}
    \item[tcp-port] Un intero non negativo che rappresenta la porta usata dal socket TCP. 
    \item[udp-port] Un intero non negativo che rappresenta la porta usata dal socket UDP.
    \item[multicast-addr] L'indirizzo di multicast sul quale i client ricevono
        le notifiche del calcolo delle ricompense.
    \item[multicast-port] La porta sulla quale i client ricevono le notifiche
        in multicast (è un intero non negativo). 
\end{description}


\end{document}